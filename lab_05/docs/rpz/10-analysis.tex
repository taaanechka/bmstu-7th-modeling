\chapter{Теоретическая часть}

Структурная схема модели приведена на рисунке \ref{img:common}.
\imgw{common}{ht!}{1\textwidth}{труктурная схема модели}

Cхема модели в терминах СМО приведена на рисунке \ref{img:smo}.
\imgw{smo}{ht!}{1\textwidth}{Cхема модели в терминах СМО}

В процессе взаимодействия клиентов с информационным центром возможны следующие режимы.
\begin{enumerate}
	\item Режим нормального обслуживания, т.е. клиент выбирает одного из
	свободных операторов, отдавая предпочтение тому, чья скорость обслуживания больше.
	\item Режим отказа в обслуживании клиента, когда все операторы заняты.
\end{enumerate}

\clearpage
\section{Переменные и уравнения имитационной модели}

Эндогенные переменные:
\begin{itemize}
	\item время обработки задания i-ым оператором; 
	\item время решения этого задания j-ым компьютером.
\end{itemize}

Экзогенные переменные:
\begin{itemize}
	\item $n_0$ --- число обслуженных клиентов; 
	\item $n_1$ --- число клиентов получивших отказ.
\end{itemize}

Уравнение имитационной модели:
\begin{equation}
	P_{отк} = \frac{n_1}{n_1 + n_0}
\end{equation}
