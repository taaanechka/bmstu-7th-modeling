\chapter{Теоретическая часть}

\section{Распределения}

\subsection{Равномерное распределение}

Случайная величина $X$ имеет равномерное распределение на отрезке $[a; b]$, если 
ее функция плотности имеет вид
\begin{equation}
	f(x) = 
	\begin{cases}
	    \frac {1}{b - a}, & a \leq x \leq b\\
	    0,  & $\text{иначе}$
	\end{cases}
\end{equation}

Обозначение: $X \sim R[a, b]$.

Функция равномерного распределения имеет следующий вид.
\begin{equation}
	F(x) = 
	\begin{cases}
		0,  & a < x \\
	    \frac {x - a}{b - a}, & a \leq x \leq b\\
	    1,  & x > b
	\end{cases}
\end{equation}

\subsection{Распределение Пуассона}

Случайная дискретная величина $X$ распределена по закону Пуассона с параметром $\lambda > 0$, 
если она принимает значения 0, 1, 2, ... с вероятностями
\begin{equation}
	P \left\{X = k\right\} = \frac {{\lambda}^k}{k!} * {e}^{- \lambda}, k \in 0, 1, 2, ...,
\end{equation}
где 

\begin{itemize}
	\item $k$ -- количество событий,
	\item $\lambda$ -- математическое ожидание случайной величины.
\end{itemize}

Обозначение: $X \sim \Pi(\lambda)$.

Функция плотности распределения имеет вид:
\begin{equation}
	P\left\{x = k\right\} = \frac {{\lambda}^k}{k!} * {e}^{- \lambda}, k \in 0, 1, 2, ...
\end{equation}

Тогда соответствующая функция распределения имеет следующий вид.
\begin{equation}
	F(x) = P\left\{X < x\right\}, X \sim \Pi(\lambda)
\end{equation}

\section{Принципы}

\subsection{Принцип $ \Delta t $}
Этот принцип заключается в последовательном анализе состояний всех блоков системы в момент $t + \Delta t $. При этом новое состояние блоков определяется в соответствии с их алгоритмическим описанием. 

\textit{Основной недостаток}: значительные затраты вычислительных ресурсов при моделировании системы. А также при недостаточно малом $\Delta t$ появляется опасность пропуска отдельных событий в системе, исключающая возможность получения правильных результатов.

\subsection{Событийный принцип}
Состояние отдельных устройств изменяется в отдельные моменты времени, совпадающие с моментами времени поступления сообщений в систему, окончания реализации задачи/процесса, возникновения прерываний, возникновения аварийных сигналов. Следовательно, моделирование и продвижение текущего времени в системе удобно проводить, используя событийный принцип.

При использовании данного принципа состояние \textbf{всех} блоков иммитационной модели анализируется лишь в момент появления какого-либо события. Момент поступления следующего события определяются минимальным значением из списка будущих событий, представляющего собой совокупность моментов ближайшего изменения состояний каждого из блоков системы.
