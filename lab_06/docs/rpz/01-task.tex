\chapter{Задание}

В билетно-кассовый РЖД центр приходят клиенты через интервал времени $0-3$
мин. 
Данный центр содержит 4 оператора: 2 терминала, причем один из терминалов --- нового формата, другой --- старого, и 2 кассы. Если все операторы заняты, клиент встает в очередь. Операторы имеют разную производительность и могут
обеспечивать обслуживание среднего запроса пользователя за $3 \pm 1$; $4 \pm 2$; $5 \pm 3$; $8 \pm 4$ мин. Клиенты стремятся занять оператора, длина очереди к которому минимальна. Полученные запросы сдаются в приемный накопитель. Откуда
выбираются на обработку. На сервере находятся компьютеры, которые обработывают данные запросы. На первый компьютер --- запросы от 1-ого
оператора, на второй --- запросы от 2-ого, на третий --- запросы от 3-его и 4-ого операторов. Время обработки запросов на 1-ом, 2-ом и 3-ем компьютерах равны соответственно 1, 2, 3 мин. 

Смоделировать процесс обработки 600 запросов.

Также в отчете необходимо сделать следующее.
\begin{enumerate}
    \item Построить структурную схему модели.
    \item Построить СМО модель.
\end{enumerate}
