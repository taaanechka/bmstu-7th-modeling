\chapter{Теоретическая часть}

\section{Равномерное распределение}

Случайная величина $X$ имеет равномерное распределение на отрезке $[a; b]$, если 
ее функция плотности имеет вид
\begin{equation}
	f(x) = 
	\begin{cases}
	    \frac {1}{b - a}, & a \leq x \leq b\\
	    0,  & $\text{иначе}$
	\end{cases}
\end{equation}

Обозначение: $X \sim R[a, b]$.

Функция равномерного распределения имеет следующий вид.
\begin{equation}
	F(x) = 
	\begin{cases}
		0,  & a < x \\
	    \frac {x - a}{b - a}, & a \leq x \leq b\\
	    1,  & x > b
	\end{cases}
\end{equation}

\section{Распределение Пуассона}

Случайная дискретная величина $X$ распределена по закону Пуассона с параметром $\lambda > 0$, 
если она принимает значения 0, 1, 2, ... с вероятностями
\begin{equation}
	P \left\{X = k\right\} = \frac {{\lambda}^k}{k!} * {e}^{- \lambda}, k \in 0, 1, 2, ...,
\end{equation}
где 

\begin{itemize}
	\item $k$ -- количество событий,
	\item $\lambda$ -- математическое ожидание случайной величины.
\end{itemize}

Обозначение: $X \sim \Pi(\lambda)$.

Функция плотности распределения имеет вид:
\begin{equation}
	P\left\{x = k\right\} = \frac {{\lambda}^k}{k!} * {e}^{- \lambda}, k \in 0, 1, 2, ...
\end{equation}

Тогда соответствующая функция распределения имеет следующий вид.
\begin{equation}
	F(x) = P\left\{X < x\right\}, X \sim \Pi(\lambda)
\end{equation}
