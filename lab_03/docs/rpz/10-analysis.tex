\chapter{Теоретическая часть}

\section{Способы генерации последовательности псевдослучайных чисел}

\subsection{Алгоритмический способ}

В качестве алгоритмического способа рассмотрим линейный конгруэнтный метод.

В данном методе каждой следующее число рассчитывается на основе предыдущего по формуле (\ref{formula1}).

\begin{equation}\label{formula1}
g_{n + 1} = (k \cdot g_n + C)\;mod\;N,\, n \geq 1
\end{equation}
где k, C -- коэффициенты, N -- модуль.

\subsection{Табличный способ}

В данном способе последовательность случайных чисел получают из заранее подготовленной таблицы (файла), данные в которой являются числами, не зависящими друг от друга.

\section{Критерий оценки}

За критерий оценки был принят критерий монотонности с опорой на критерий $\chi^2$.

\subsection{Критерий $\chi^2$}

Данный критерий относится к самым известным из статистических критериев и является основным методом, который используют в сочетании с другими критериями.

Критерий $\chi^2$ позволяет выяснить, удовлетворяет ли генератор случайных чисел требованию равномерного распределения.

Используется статистика, представленная формулой (\ref{formula2}).

\begin{equation}\label{formula2}
	V = \frac{1}{n}\sum_{s=1}^{k} \left( \frac{Y_s^2}{p_s} \right) - n
\end{equation}
где n -- количество независимых испытаний, k -- количество категорий, $Y_s$ -- число наблюдений, которые относятся к категории $S$, $p_s$ -- вероятность того, что каждое наблюдение относится к категории $S$.

\subsection{Критерий монотонности}

Данный критерий используется для проверки распределения длин монотонных
подпоследовательностей в последовательностях вещественных чисел.

Рассмотрим следующий пример. 

Пусть дана выборка:

0.7, 0.03, 0.4, 0.17, 0.24, 0.55, 0.33, 0.64

Найдем в ней отрезки возрастания при условии, что смежные отрезки не являются независимыми, а значит, необходимо <<выбросить>> элемент, который следует непосредственно за серией. Таким образом,
когда $ X_j $ больше $ X_j+1 $, начнем следующую серию с $ X_{j+2} $.

В данной последовательности найдено 4 отрезка возрастания:

[0.7], [0.4], [0.24, 0.55], [0.64].

Таким образом, в данной последовательности имеется три отрезка длиной 1 и один отрезок длиной 2.

В таком случае, после подсчета количества отрезков возрастания с
различной длиной, можем использовать критерий $\chi^2$ со следующими вероятностями:

\begin{equation}\label{formula3}
	\begin{cases}
		p_r = \frac {1}{r!} - \frac {1}{(r+1)!}, & r < t\\
		p_t = \frac {1}{t!}, & r \geq t\\
	\end{cases}
\end{equation}


